%
% tests.tex
%
% Copyright (C) 2021 by SpaceLab.
%
% FloripaSat-2 Documentation
%
% This work is licensed under the Creative Commons Attribution-ShareAlike 4.0
% International License. To view a copy of this license,
% visit http://creativecommons.org/licenses/by-sa/4.0/.
%

%
% \brief Test plan and results.
%
% \author Gabriel Mariano Marcelino <gabriel.mm8@gmail.com>
%
% \institution Universidade Federal de Santa Catarina (UFSC)
%
% \version 0.1.0
%
% \date 2021/01/21
%

\chapter{Test Plan and Results} \label{ch:test-plan}

The FloripaSat-2 test plan is structured into four phases: Module tests, FlatSat, Engineering integration, and Flight integration. This plan is summarized in the \autoref{tab:test-plan} and includes the components under test for each phase. 

\begin{table}[!h]
    \begin{center}
        \begin{tabular}{ll}
            \toprule[1.5pt]
            \textbf{Phase}    & \textbf{Components}       \\
            \midrule
            Module tests             & OBDH module \\
                                     & EPS module \\
                                     & TTC module \\
                                     & BATC4 board \\
                                     & IIP boards \\
                                     & PC104-ADPT boards \\
                                     & ACS components (simulation) \\
                                     & Mechanical (CAD assessment) \\
            \midrule
            FlatSat                  & Satellite core (OBDH+EPS+TTC+BATC4) \\
                                     & Satellite core + GRS \\
                                     & Satellite core + Payloads \\
                                     & Satellite core + Payloads + GRS \\
                                     & Satellite (long-term evaluation) \\
            \midrule
            Engineering integration  & Mechanical assembly (repeated when required) \\
            (clean room preferable)  & Satellite core + Payloads + GRS \\
                                     & Satellite (long-term evaluation) \\
                                     & Satellite + Solar Panels \\
                                     & Preliminary environmental tests \\ 
            \midrule
            Flight integration       & Mechanical assembly (flight components) \\
            (clean room mandatory)   & Satellite (short-term evaluation) \\
                                     & Satellite + Antenna (deployment) \\
                                     & Mechanical reassembly for flight \\
                                     & Satellite (long-term evaluation) \\
                                     & Qualification environmental tests \\ 
            \bottomrule[1.5pt]
        \end{tabular}
        \caption{Test plan phases and tested components.}
        \label{tab:test-plan}
    \end{center}
\end{table}

This section focus on providing an overview of the planned testing workflow and a description of the strategies approached to accomplish the mission objectives. The module tests focus on the individual modules operation and behavior, in which a general template is provided in this document and each module applies it for their needs. The FlatSat phase is the first modules integration in a debug platform to validate the system from a development perspective (described with more details in \cite{flatsat}). Finally, the engineering integration is the final development campaign aiming to validate the system from a mission perspective and the flight integration is the actual CubeSat assembly using the flight components and final assessments to prepare the satellite for launch. The integration details, procedures and qualification proccess are described with more depth in the \autoref{ch:ait}. % maybe it will be splitted in a separated document

\section{Module tests}

The first phase is the foundation for the satellite, consolidating the base design for each subsystem and shaping their relations. Therefore, several techniques were employed to ensure a solid test strategy: several inspections of the boards design and manufacturing quality; manual experimental assessments of various hardware electrical, mechanical and behavioral parameters; remotely automated tests using a continuos integration (CI)\nomenclature{\textbf{CI}}{\textit{Continuos Integration.}} approach; semi-automated tests using a hardware-in-the-loop (HIL)\nomenclature{\textbf{HIL}}{\textit{Hardware-In-the-Loop.}} strategy; simulations; and CAD models assessment.

\subsection{Workflow}

The following topics lists the template workflow used to create the procedures for each subsystem. Each module documentation has its own test chapter describing the process in detail, from procedures to success criteria.

% \item [\footnotesize{XX-00}] Text 
\subsubsection{Visual Inspection} 
\begin{enumerate} \setlength\itemsep{-0.3em}
    \item Packaging quality assessment
    \item Board manufacturing and assembly quality
    \item 3D model comparison
    \item Layers marker
    \item Labels (schematics comparison) 
    \item High resolution photos for documentation
\end{enumerate}

\subsubsection{Mechanical Inspection}
\begin{enumerate} \setlength\itemsep{-0.3em}
    \item Board dimensions and mounting holes positioning
    \item Board weight measurement
\end{enumerate}

\subsubsection{Integration Inspection}
\begin{enumerate} \setlength\itemsep{-0.3em}
    \item Check connectors pinout against the documentation (not schematics)
    \item Check connectors positioning (if applicable)
\end{enumerate}

\subsubsection{Electrical Inspection}
\begin{enumerate} \setlength\itemsep{-0.3em}
    \item Solder shorts
    \item Missing components
    \item Lifted pins
    \item Poor soldering
    \item Swapped components
    \item Components partnumber
    \item Components polarity (schematic comparison)
    \item Components defined to not be soldered (DNP)
\end{enumerate}

\subsubsection{Electrical Testing}
\begin{enumerate} \setlength\itemsep{-0.3em}
    \item Continuity test
    \item Power up procedures (check LEDs and testpoints)
    \item Average input power consumption measurement
    \item Average output power source measurement (if applicable) 
    \item Power tracks temperature (if applicable)
    \item Simple signal integrity (if applicable)
\end{enumerate}

\subsubsection{Functional Testing}
\begin{enumerate} \setlength\itemsep{-0.3em}
    \item Run a simple test code (if applicable) 
    \item Run the system code (if applicable and available) 
    \item Check the system hardware self-test flags (if applicable and available) 
    \item Monitor basic LEDs behavior (if applicable) 
    \item Monitor the debug serial port logs (if applicable)
\end{enumerate}

\subsubsection{Module Testing}
\begin{enumerate} \setlength\itemsep{-0.3em}
    \item Run simulations and review results (if applicable)
    \item Review operation behavior against the documentation (if applicable)
    \item Review features and requirements fulfillment
    \item Review communication buses configuration and protocol (if applicable)
    \item Review data packages, power buses and control signals
    \item Review and evaluate operation edge cases
    \item Run remote automated code tests (if applicable)
    \item Run system test codes in the board (if applicable)
    \item Run latest stable code version, monitor logs and qualify behavior (if applicable)
\end{enumerate}


\subsection{Continuos Integration}

In order to detect errors and bugs in the early stages of development, a continuos integration workflow was setup for automated firmware tests focusing in small scope verfications (i.e., unit tests). Instead of executing the code in the target processor, the tests are executed remotely in a host computer through the usage of an unit testing framework, called "cmocka" \textcolor{red}{cite cmocka}. This tool allows to abstract the inherent hardware dependencies of embedded systems to enable firmware tests without errors introduced by hardware problems (exection in a consolidated platform, the computer), which provides an optimal behavioral assessment of the code implementations. This approach not only support remote testing, but promote continuos test execution, which is essencial to detect erros and architectural issues. The integration of these procedures is powered by "GitHub Actions" \textcolor{red}{cite GH Action}, which provides a host machine and a dashboard inside the same environment of the already used version control, source distribution and management tool.

The unit tests follows a layered structure accordingly with the firmware layers. This is used alongside mockups (i.e., interfaces that abstract what the layer receive as input without having to implement the underlying functionality), which allows independency between the layers and abstract the actual hardware dependencies with an emulated behavior.


\subsection{Hardware-In-the-Loop}



\subsubsection{Integration Tests}

\begin{itemize} \setlength\itemsep{-0.3em}
    \item Operating system initialization: assert memory allocation (RAM, stack, heap), hooks and etc;
    \item Boot sequence (as similar to the actual procedure as possible).
    \item Operating system task/queue/interrupts priority, constraints, size, depth and delay checks: use dummy task/queue/interrupts (same config as actual system).
    \item Short-term system check: after 1 hour, exit without error logs.
    \item Mid-term system check: after 1 day, exit without error logs.
    \item Long-term system check (used in flatsat): after 1 week, exit without flatsat/integration error logs.
\end{itemize}

\subsubsection{Workflow}

\begin{itemize} \setlength\itemsep{-0.3em}
    \item Always it is a build->flash->test, change main and repeat.
    \item It must have a test folder containing subfolders (hardware, drivers, devices, app, integration) and a json file (with name, path and type).
    \item Inside the workflow is called a python script that read this json and setup variables to allow running multiple main file swaps for each test type.
    \item There are 5 different workflows, one for each test type: hardware, drivers, devices, app, integration;
    \item The workflow, tests and scripts must be reviewed before each release.
    \item Idea: for short/mid/long-term tests, the workflow should evaluate the log messages offline instead of real time, in which a job is scheduled to run just after this period and ``a script'' will read the log file and search for the test criteria, giving the actual CI result.
    \item Idea: Inside the code, using the log message approach, we might create our ultra lightweight framework that consists of only log types (colors) and log messages (specific strings). This way we do not modify our current workflow and we can add a simple scheme to access the flight code.
    \item Unit Tests = Tests performed per firmware unit.
    \item Integration Tests = Tests performed per firmware component (several units abstracted).
\end{itemize}



\section{Flatsat}

To test all modules during the development of the projet, a flatsat platform was developed. The FlatSat Platform is a testbed for CubeSat PCB modules. FlatSats enable easier, faster and a secure method for testing subsystens independently while been integrated in a flat design before going to integration on a CubeSat form factor. The PCB can support up to 7 modules, all PC-104 pins are interligated to flexibilize its use, only the particularity connection between modules need to be be taken into account. One PC-104 has inverted pinout, the board also makes it possible to have two seperate power supplies, a UART to USB converter for 4 modules, kill-switches activation though SPDTs, Remove Before Flight (RBF) pin header, connector for charging batteries and SMA connectors for antennas. A picture of the flatsat board can be seen in \autoref{fig:flatsat-top}.

\begin{figure}[!ht]
    \begin{center}
        \includegraphics[width=\textwidth]{figures/flatsat_top_image}
        \caption{Top view of the flatsat board.}
        \label{fig:flatsat-top}
    \end{center}
\end{figure}

More information about the Flatsat Platform can be found in \cite{flatsat}.




\subsection{Environmental Tests}

LIT\nomenclature{\textbf{LIT}}{\textit{Laboratório de Integração e Testes.}}

\cite{marcelino2021}

\subsubsection{Mass Verification}

This test checks the total mass of the satellite (without RBF tag), which must be less than 2,66 kg \cite{cds}. The verification is made with a precision balance. \autoref{fig:mass-verification} examplifies this process with FloripaSat-I total mass.

\begin{figure}[!ht]
    \begin{center}
        \includegraphics[width=0.5\textwidth]{figures/mass-test}
        \caption{Mass verificatiton of FloripaSat-I.}
        \label{fig:mass-verification}
    \end{center}
\end{figure}

\subsubsection{Center of Gravity}

This test checks the center of gravity (CG) of the satellite, which must be less than 2 cm from the geometric center (see \autoref{fig:cg}) \cite{cds}. To perform this test, a simple test-bench based on two parallel bars fixed on a plate (4 cm from each other) can be used. The geometric center of the satellite is put in the middle of the bars and, if the satellite does not fall, the CG is within the radius of 2 cm. This strategy does not measure the location of CG, however, it does prove if the satellite follows the requirement.

\begin{figure}[!htb]
    \begin{center}
        \subfigure[$X$ axis.\label{fig:fsat-fm-x-axis}]{\includegraphics[width=0.2\textwidth]{figures/fsat_fm_x_axis.png}}
        ~
        \subfigure[$Y$ axis.\label{fig:fsat-fm-y-axis}]{\includegraphics[width=0.2\textwidth]{figures/fsat_fm_y_axis.png}}
        ~
        \subfigure[$Z$ axis.\label{fig:fsat-fm-z-axis}]{\includegraphics[width=0.2\textwidth]{figures/fsat_fm_z_axis.png}}
        \caption{Center of gravity of FloripaSat-I within 2 cm from geometric center.}
        \label{fig:cg}
    \end{center}
\end{figure}

\subsubsection{Vibration Test}

To measure and control the acceleration profile during the dynamic tests, accelerometers should be positioned on three external surfaces of the satellite, one on each axis, over areas without solar cells. The satellite should be fixed on a shaker. Figure \ref{fig:fsat-vibration-accel} shows some of the accelerometers and Figure \ref{fig:fsat-shaker} shows the satellite during a vibration test.

\begin{figure}[!htb]
    \begin{center}
        \subfigure[Position of the accelerometers.\label{fig:fsat-vibration-accel}]{\includegraphics[height=4.5cm]{figures/fsat_fm_accel.jpg}}
        ~
        \subfigure[Shaker.\label{fig:fsat-shaker}]{\includegraphics[height=4.5cm]{figures/fsat_fm_shaker.jpg}}
        \caption{Vibration test.}
        \label{fig:vibration-test}
    \end{center}
\end{figure}

The CubeSat should be tested entirely off, with RBF pin removed but with the Kill-Switches pressed, in a 2U Test POD, simulating the normal launching condition. The set of vibration tests follows \autoref{fig:vibration_procedure}.

\begin{figure}[!ht]
    \begin{center}
        \includegraphics[width=0.6\textwidth]{figures/vibration_procedure.pdf}
        \caption{Sequence of dynamic tests.}
        \label{fig:vibration_procedure}
    \end{center}
\end{figure}

A signature testing should be conducted before and after the tests (sinusoidal and random vibration), in order to identify the presence of significant variations in the dynamic response, a condition that may represent mechanical failures. For the signature task, \autoref{tab:vibration-test-dynamic-1} presents the specifications.

\begin{table}[!h]
    \begin{center}
        \begin{tabular}{ll}
            \toprule[1.5pt]
            \textbf{Name}    & \textbf{Parameter}       \\
            \midrule
            Test method      & Sinusoidal sweep testing \\
            Frequency range  & 5 - 2000 Hz              \\
            Vibration level  & 0,25 g                   \\
            Sweep rate       & 2 octaves per minute     \\
            Number of sweeps & 1 (5 - 2000 Hz)          \\
            Test axes        & 3 ($X$, $Y$, $Z$)        \\
            \bottomrule[1.5pt]
        \end{tabular}
        \caption{Resonance survey test (signature).}
        \label{tab:vibration-test-dynamic-1}
    \end{center}
\end{table}

%Regarding the sinusoidal sweeping vibration, Table 3 brings the envelope of the test, and so does Fig. 24 in a graphic format.

\begin{figure}[!ht]
    \begin{center}
        \includegraphics[width=\textwidth]{curves/sine_test.pdf}
        \caption{Sinusoidal sweeping vibration curve.}
        \label{fig:vibration-sinusoidal-curve}
    \end{center}
\end{figure}

\begin{figure}[!ht]
    \begin{center}
        \includegraphics[width=\textwidth]{curves/random_vibration.pdf}
        \caption{Random vibration curve.}
        \label{fig:vibration-test}
    \end{center}
\end{figure}

\subsubsection{Thermal Test}

For the thermal tests, thermocouples should be attached on different points on the surface of the satellite, including over the solar panels and structure. As an example, \autoref{fsat-thermal-test} shows FloripaSat-I ready for thermal tests. The parameters of the tests are indicated in \autoref{tab:fsat-thermal-cycling}.

\begin{figure}[!ht]
    \begin{center}
        \includegraphics[width=0.5\textwidth]{figures/fsat_fm_thermal_cycling.jpg}
        \caption{FloripaSat-I during the thermal cycling (with thermocouples).}
        \label{fig:fsat-thermal-test}
    \end{center}
\end{figure}

\begin{table}[!h]
    \begin{center}
        \begin{tabular}{llll}
            \toprule[1.5pt]
            \multicolumn{2}{c}{\textbf{Thermal cycle}}  & \multicolumn{2}{c}{\textbf{Bake out}}        \\
            \midrule
            \textbf{Parameter}     & \textbf{Value}     & \textbf{Parameter} & \textbf{Value}          \\
            \midrule
            Number of cycles       & 2                  & \multicolumn{2}{c}{Part 1}                   \\
            \cmidrule{3-4}
            Min. temp. ($T_{min}$) & -15 $^\circ$C      & Pressure           & <1$\times 10^{-4}$ mbar \\
            Max. temp. ($T_{max}$) & +50 $^\circ$C      & Temperature        & 23 $^\circ$C            \\
            Duration in $T_{min}$  & 30 min             & Duration           & 12 hours                \\
            \cmidrule{3-4}
            Duration in $T_{max}$  & 60 min             & \multicolumn{2}{c}{Part 2}                   \\
            \cmidrule{3-4}
            Heating rate           & 5.5 $^\circ$C/min  & Pressure           & <1$\times 10^{-4}$ mbar \\
            Cooling rate           & 3.5 $^\circ$C/min  & Temperature        & 60 $^\circ$C            \\
            Stabilization criteria & 1 $^\circ$C/10 min & Duration           & 6 hours                 \\
            \bottomrule[1.5pt]
        \end{tabular}
        \caption{Parameters for the bake out and thermal cycling.}
        \label{tab:fsat-thermal-cycling}
    \end{center}
\end{table}



\section{Preliminary Results}


\subsubsection{Output Power of the Radio Modules}

The output power of the radio modules can be measured using a spectrum analyzer, as can be seen in the picture of \autoref{fig:rf-output-power-test}.

\begin{figure}[!ht]
    \begin{center}
        \includegraphics[width=\textwidth]{figures/rf-output-power-test.jpg}
        \caption{RF output power test with the radio modules connected to a spectrum analyzer.}
        \label{fig:rf-output-power-test}
    \end{center}
\end{figure}

The measured values for the beacon and downlink transmitters are available in Figures \ref{fig:beacon-power} and \ref{fig:downlink-power} respectively.

\begin{figure}[!ht]
    \begin{center}
        \includegraphics[width=\textwidth]{curves/beacon_output_power.pdf}
        \caption{Output power of the beacon radio.}
        \label{fig:beacon-power}
    \end{center}
\end{figure}

\begin{figure}[!ht]
    \begin{center}
        \includegraphics[width=\textwidth]{curves/downlink_output_power.pdf}
        \caption{Output power of the downlink radio.}
        \label{fig:downlink-power}
    \end{center}
\end{figure}
