%
% ground_segment.tex
%
% Copyright (C) 2021 by SpaceLab.
%
% GOLDS-UFSC Documentation
%
% This work is licensed under the Creative Commons Attribution-ShareAlike 4.0
% International License. To view a copy of this license,
% visit http://creativecommons.org/licenses/by-sa/4.0/.
%

%
% \brief Ground segment chapter.
%
% \author Gabriel Mariano Marcelino <gabriel.mm8@gmail.com>
%
% \institution Universidade Federal de Santa Catarina (UFSC)
%
% \version 0.1.0
%
% \date 2020/06/06
%

\chapter{Ground Segment} \label{ch:ground-segment}

\section{Hardware}

\subsection{Antennas}

There are two antennas in the ground station: One for VHF and one for the UHF band. The main characteristics of these antennas can be seen in \autoref{tab:grs-antennas}

\begin{table}[ht]
    \centering
    \begin{tabular}{lccc}
        \toprule[1.5pt]
        \textbf{Characteristic} & \textbf{VHF Antenna}  & \textbf{UHF Antenna}  & \textbf{Unit} \\
        \midrule
        Brand                   & M$^{2}$               & Cushcraft             & - \\
        Model                   & 2MCP14                & A719B                 & - \\
        Type                    & Yagi                  & Yagi                  & - \\
        Number of elements      & 14                    & 19                    & - \\
        Frequency range         & 143-148               & 430-450               & MHz \\
        Gain                    & 12,34                 & 15,5                  & dBi \\
        Power rating            & 1500                  & 2000                  & W \\
        Boom length             & 3,2                   & 4,1                   & m \\
        Longest element         & 1,02                  & 0,34                  & m \\
        Weight                  & 2,72                  & 2,55                  & kg \\
        \bottomrule[1.5pt]
    \end{tabular}
    \caption{Main characteristics of the ground segment antennas.}
    \label{tab:grs-antennas}
\end{table}

More information about the VHF and UHF antennas can be found in \cite{2mcp14} and \cite{a719b} respectively.

\subsubsection{Surge Protector}

.

\subsection{Rotators}

Both antennas (VHF and UHF) track the satellite through a two axis rotator (azimuth and elevation). The used model is the Yaesu G-5500, which provides 450$^{\circ}$ azimuth and 180$^{\circ}$ elevation control of medium and large size unidirectional satellite antenna arrays under remote control from station operation position.

A picture of the G-5500 rotator (and controller) can be seen in \autoref{fig:g5500}, the main characteristics can be found in \autoref{tab:grs-rotor}.

\begin{figure}[!ht]
    \begin{center}
        \includegraphics[width=0.6\textwidth]{figures/g5500.jpg}
        \caption{Yaesu G-5500 rotator and controller.}
        \label{fig:g5500}
    \end{center}
\end{figure}

\begin{table}[ht]
    \centering
    \begin{tabular}{lcc}
        \toprule[1.5pt]
        \textbf{Characteristic}                     & \textbf{Value}        & \textbf{Unit} \\
        \midrule
        Brand                                       & Yaesu                 & - \\
        Model                                       & G-5500                & - \\
        Voltage requirement                         & 110-120 or 200-240    & $V_{AC}$ \\
        Motor voltage                               & 24                    & V$_{AC}$ \\
        Rotation time (elevation, 180$^{\circ}$)    & 67                    & s \\
        Rotation time (azimuth, 360$^{\circ}$)      & 58                    & s \\
        Maximum continuous operation                & 5                     & min \\
        Rotation torque (elevation)                 & 14                    & kg-m \\
        Rotation torque (azimuth)                   & 6                     & kg-m \\
        Braking torque (elevation and azimuth)      & 40                    & kg-m \\
        Vertical load                               & 200                   & kg \\
        Pointing accuracy                           & $\pm$ 4               & \% \\
        Wind surface area                           & 1                     & $m^{2}$ \\
        Weight (rotator)                            & 9                     & kg \\
        Weight (controller)                         & 3                     & kg \\
        \bottomrule[1.5pt]
    \end{tabular}
    \caption{Main characteristics of antennas' rotators.}
    \label{tab:grs-rotor}
\end{table}

More information about the ground station rotator can be found in \cite{g5500}.

\subsection{Amplifiers}

\subsubsection{Power Amplifier}

PA\nomenclature{\textbf{PA}}{\textit{Power Amplifier}}...

\subsubsection{Low Noise Amplifiers}

LNA\nomenclature{\textbf{LNA}}{\textit{Low Noise Amplifier}}...

\subsection{Radios}

.

\subsection{Processing and Control}

.

\section{Satellite Tracking}

To track the satellite and for orbit prediction, the GPredict software \cite{gpredict} will be used. Gpredict is a real-time satellite tracking and orbit prediction application. It can track a large number of satellites and display their position and other data in lists, tables, maps, and polar plots (radar view). Gpredict can also predict the time of future passes for a satellite, and provide you with detailed information about each pass. Gpredict is free software licensed under the GNU General Public License. A picture of the main window of GPredict can be seen in \autoref{fig:gpredict}.

\begin{figure}[!ht]
    \begin{center}
        \includegraphics[width=\textwidth]{figures/gpredict.png}
        \caption{Main window of GPredict.}
        \label{fig:gpredict}
    \end{center}
\end{figure}

\section{Packet Decoding}

\cite{spacelab-decoder}

\begin{figure}[!ht]
    \begin{center}
        \includegraphics[width=\textwidth]{figures/spacelab-decoder.png}
        \caption{Main window of the SpaceLab Decoder application.}
        \label{fig:spacelab-decoder}
    \end{center}
\end{figure}

\section{PCDs}

PCD\nomenclature{\textbf{PCD}}{\textit{``Plataforma de Coleta de Dados'', or Data Collection Platform}}...
