%
% overall.tex
%
% Copyright (C) 2021 by SpaceLab.
%
% GOLDS-UFSC Documentation
%
% This work is licensed under the Creative Commons Attribution-ShareAlike 4.0
% International License. To view a copy of this license,
% visit http://creativecommons.org/licenses/by-sa/4.0/.
%

%
% \brief Overall description chapter.
%
% \author Gabriel Mariano Marcelino <gabriel.mm8@gmail.com>
%
% \institution Universidade Federal de Santa Catarina (UFSC)
%
% \version 0.1.0
%
% \date 2020/06/05
%

\chapter{Overall Description} \label{ch:overall}

.

\section{General Diagrams}

.

\section{General Behaviour}

.

\section{Orbit Parameters}

.

\section{Power Budget}

.

\section{Link Budget}

\subsection{Distance to Satellite at Horizon}

\begin{itemize}
    \item $R_{e}$ = Earth radius = 6378 km
    \item $h$ = Satellite altitude = 600 km
    \item $d$ = Distance to sattellite at horizon
\end{itemize}

\begin{equation}
d = \sqrt{2\cdot R_{e}\cdot h + h^{2}} = \sqrt{2\cdot 6378\cdot 600 + 600^{2}} = \mathbf{2830\ km}
\end{equation}


%\section{Power at Receiver}

%\begin{equation}
%P_{r} = P_{t} + G_{t} + G_{r} - L_{p}\ [dB]
%\end{equation}


\subsection{VHF Link}

\begin{itemize}
    \item Direction: \textbf{Downlink}
    \item Frequency ($f$): \textbf{145,97 MHz}
    \item Modulation: \textbf{MSK}
    \item Datarate ($baud$): \textbf{1200 bps}
    \item Output Power ($P_{t}$): \textbf{30 dBm (1 W)}
    \item Protocol: \textbf{NGHam (with FEC)}
    \item Satellite antenna gain ($G_{t}$): \textbf{0 dB}
    \item Groundstation antenna gain ($G_{r}$): \textbf{15 dB}
\end{itemize}

\subsubsection{Free-Space Path Loss}

\begin{equation}
FSPL = \left( \frac{4\pi d f}{c} \right)^{2}
\end{equation}

\begin{equation}
    \begin{split}
        FSPL^{dB} & = 20\log\left(\frac{4\pi}{c}\right) + 20\log\left(d\right) + 20\log\left(f\right) \\
                  & = 32,45 + 20\log\left(\frac{d}{1\ km}\right) + 20\log\left(\frac{f}{1\ MHz}\right) \\
    \end{split}
\end{equation}

\begin{equation}
FSPL^{dB}_{max} = 32,45 + 20\log\left(\frac{2830}{1\ km}\right) + 20\log\left(\frac{145,97}{1\ MHz}\right) = \mathbf{144,8\ dB}
\end{equation}

\begin{equation}
FSPL^{dB}_{min} = 32,45 + 20\log\left(\frac{600}{1\ km}\right) + 20\log\left(\frac{145,97}{1\ MHz}\right) = \mathbf{131,3\ dB}
\end{equation}

\begin{equation}
\mathbf{131,3 \leq FSPL^{dB} \leq 144,8\ dB}
\end{equation}

\subsubsection{Signal-to-Noise-Ratio}

The Signal-to-Noise-Ratio (SNR\nomenclature{\textbf{SNR}}{\textit{Signal To Noise Ratio}}) of the beacon signal can be expressed using the \autoref{eq:snr}:

\begin{equation} \label{eq:snr}
SNR = \frac{E_{b}}{N_{0}} = \frac{P_{t}G_{t}G_{r}}{kT_{s}RL_{p}}
\end{equation}

Where:

\begin{itemize}
    \item $P_{t}$ = Transmitter power
    \item $G_{t}$ = Transmitter gain
    \item $G_{r}$ = Receiver gain
    \item $k$ = Boltzmann's constant ($\cong 1,3806 \times 10^{-23}\ J/K$)
    \item $T_{s}$ = System noise temperature
    \item $R$ = Data rate in bits per seconds (bps)
    \item $L_{p}$ = Free-Space Path Loss (FSPL)
\end{itemize}

The SNR value in decibels can be calculated using the \autoref{eq:snr-db}:

\begin{equation} \label{eq:snr-db}
    \begin{split}
        SNR^{dB} & = 10\log_{10}\left( \frac{E_{b}}{N_{0}} \right) = 10\log_{10} \left( \frac{P_{t}G_{t}G_{r}}{kT_{s}RL_{p}} \right) \\
                 & = P_{t}^{dBm} - 30 + G_{t}^{dBi} + G_{r}^{dBi} - L_{p}^{dB} - 10\log k - 10\log T_{s} - 10\log R
    \end{split}
\end{equation}

\begin{equation}
SNR^{dB}_{max} = 30 - 30 + 0 + 15 - 131,3 - = dB
\end{equation}

\begin{equation}
SNR^{dB}_{min} = 30 - 30 + 0 + 15 - 144,8 - = dB
\end{equation}

\begin{equation}
\mathbf{ \leq SNR^{dB} \leq \ dB}
\end{equation}

\subsection{UHF Links}

\subsubsection{Main UHF Link}

\begin{itemize}
    \item Direction: \textbf{Downlink and uplink}
    \item Frequency: \textbf{436,9 MHz}
    \item Modulation: \textbf{MSK}
    \item Datarate: \textbf{4800 bps}
    \item Output power: \textbf{30 dBm (1 W)}
    \item Protocol: \textbf{NGHam (with FEC)}
    \item Satellite antenna gain ($G_{r}$): \textbf{0 dB}
    \item Groundstation antenna gain ($G_{t}$): \textbf{18 dB}
\end{itemize}

\begin{equation}
FSPL_{max} = 32,45 + 20\cdot \log\left(\frac{2830}{1\ km}\right) + 20\cdot \log\left(\frac{436,9}{1\ MHz}\right) = \mathbf{154,3\ dB}
\end{equation}

\begin{equation}
FSPL_{min} = 32,45 + 20\cdot \log\left(\frac{600}{1\ km}\right) + 20\cdot \log\left(\frac{436,9}{1\ MHz}\right) = \mathbf{140,8\ dB}
\end{equation}

\begin{equation}
\mathbf{140,8 \leq L_{p} \leq 154,3\ dB}
\end{equation}

\subsubsection{EDC UHF Link}

\begin{itemize}
    \item Direction: \textbf{Uplink}
    \item Frequency: \textbf{401.635 MHz}
    \item Modulation: \textbf{BPSK}
    \item Datarate: \textbf{400 bps}
    \item Protocol: \textbf{SBCD}
    \item Satellite antenna gain ($G_{r}$): \textbf{0 dB}
    \item Groundstation antenna gain ($G_{t}$): \textbf{XX dB}
\end{itemize}

\subsection{Link Budget Analysis}

\begin{table}[!h]
    \centering
    \begin{tabular}{lccccc}
        \toprule[1.5pt]
        \textbf{Variable} & \textbf{Beacon} & \textbf{Downlink} & \textbf{Uplink} & \textbf{Uplink (Payload)} & \textbf{Unit}\\
        \midrule
        Frequency                       & 145,97    & 436,9     & 436,9     & 401,635   & MHz \\
        Transmit power                  & 30        & 30        & 47*       & ??        & dBm \\
        Transmitter loss                & ??        & ??        & ??        & ??        & dB \\
        FSPL                            & 144,8     & 154,3     & 154,3     & ??        & dB \\
        Other losses                    & ??        & ??        & ??        & ??        & ?? \\
        Receive antenna gain            & 15*       & 18*       & 0         & 0         & dBi \\
        Receiver noise temp.            &           &           &           &           & K \\
        Antenna noise temp.             &           &           &           &           & K \\
        System noise temp.              &           &           &           &           & K \\
        Data rate                       & 1200      & 4800      & 4800      & 400       & bps \\
        Received SNR                    & ??        & ??        & ??        & ??        & dB \\
        SNR required for $10^{-5}$ BER  & ??        & ??        & ??        & ??        & dB \\
        Link margin                     & $\leq$ ?? & $\leq$ ?? & $\leq$ ?? & $\leq$ ?? & dB \\
        \bottomrule[1.5pt]
    \end{tabular}
    \caption{Link budget results.}
    \label{tab:link-budget-results}
\end{table}

\section{PC-104 Bus}

\begin{figure}[!ht]
    \begin{center}
        \includegraphics[width=0.5\textwidth]{figures/pc104-diagram}
        \label{fig:pc104-diagram}
        \caption{Reference diagram of the PC-104 bus.}
    \end{center}
\end{figure}

\begin{table}[!h]
    \centering
    \begin{tabular}{cllll}
        \toprule[1.5pt]
        \textbf{Pin Row}   & \textbf{H1 Odd}  & \textbf{H1 Even} & \textbf{H2 Odd} & \textbf{H2 Even} \\
        \midrule
        1-2                & -                & -                & -               & -                \\
        3-4                & -                & -                & EDC\_1\_EN      & EDC\_2\_EN       \\
        5-6                & -                & -                & BE\_UART\_RX    & -                \\
        7-8                & RA\_GPIO\_0      & RA\_GPIO\_1      & BE\_UART\_TX    & GPIO\_0          \\
        9-10               & RA\_GPIO\_2      & -                & -               & -                \\
        11-12              & RA\_RESET        & RA\_EN           & BE\_SPI\_MOSI   & BE\_SPI\_CLK     \\
        13-14              & -                & -                & BE\_SPI\_CS     & BE\_SPI\_MISO    \\
        15-16              & -                & -                & -               & -                \\
        17-18              & EDC\_UART\_RX/TX & PLX\_EN          & -               & GPIO\_1          \\
        19-20              & EDC\_UART\_TX/RX & GPIO\_2          & -               & GPIO\_3          \\
        21-22              & -                & -                & -               & GPIO\_4          \\
        23-24              & -                & -                & -               & -                \\
        25-26              & -                & -                & -               & -                \\
        27-28              & -                & -                & -               & -                \\
        29-30              & GND              & GND              & GND             & GND              \\
        31-32              & GND              & GND              & GND             & GND              \\
        33-34              & -                & -                & -               & -                \\
        35-36              & RD\_SPI\_CLK     & -                & ANT\_VCC        & ANT\_VCC         \\
        37-38              & RD\_SPI\_MISO    & -                & -               & -                \\
        39-40              & RD\_SPI\_MOSI    & RD\_SPI\_CS      & -               & -                \\
        41-42              & PL\_I2C\_SDA     & -                & -               & GPIO\_5          \\
        43-44              & PL\_I2C\_SCL     & -                & -               & -                \\
        45-46              & OBDH\_VCC        & OBDH\_VCC        & BAT\_VCC        & BAT\_VCC         \\
        47-48              & EDC\_VCC         & EDC\_VCC         & -               & -                \\
        49-50              & RD\_VCC          & RD\_VCC          & EPS\_I2C\_SDA   & -                \\
        51-52              & BE\_VCC          & BE\_VCC          & EPS\_I2C\_SCL   & -                \\
        \bottomrule[1.5pt]
    \end{tabular}
    \caption{PC-104 bus pinout.}
    \label{tab:pc104-pinout}
\end{table}

\begin{table}[!h]
    \centering
    \begin{tabular}{lL{0.2\textwidth}L{0.15\textwidth}L{0.33\textwidth}}
        \toprule[1.5pt]
        \textbf{Signal}  & \textbf{Pin(s)} & \textbf{Used By}     & \textbf{Description} \\
        \midrule
        GND              & H1-29, H1-30, H1-31, H1-32, H2-29, H2-30, H2-31, H2-32 & All                  & Ground reference \\
        BAT\_VCC         & H2-45, H2-46    & EPS                  & Battery terminals (+) \\
        ANT\_VCC         & H2-35, H2-36    & EPS, ANT             & Antenna power supply (3.3 V) \\
        OBDH\_VCC        & H1-45, H1-46    & EPS, OBDH            & OBDH power supply (3.3 V) \\
        EDC\_VCC         & H1-47, H1-48    & EPS, EDC 1, EDC 2    & EDC power supply (5 V) \\
        RD\_VCC          & H1-49, H1-50    & EPS, TTC             & Main radio power supply (5 V) \\
        BE\_VCC          & H1-51, H1-52    & EPS, TTC             & Beacon power supply (6 V) \\
        RD\_SPI\_CLK     & H1-35           & OBDH, TTC            & CLK signal of the main radio SPI bus \\
        RD\_SPI\_MISO    & H1-37           & OBDH, TTC            & MISO signal of the main radio SPI bus \\
        RD\_SPI\_MOSI    & H1-39           & OBDH, TTC            & MOS signal of the main radio SPI bus \\
        RD\_SPI\_CS      & H1-40           & OBDH, TTC            & CS signal of the main radio SPI bus \\
        EPS\_I2C\_SDA    & H2-49           & OBDH, EPS            & SDA signal of the EPS I2C bus \\
        EPS\_I2C\_SCL    & H2-51           & OBDH, EPS            & SCL signal of the EPS I2C bus \\
        BE\_UART\_RX     & H2-5            & EPS, TTC             & EPS TX, Beacon RX (UART bus) \\
        BE\_UART\_TX     & H2-7            & EPS, TTC             & EPS RX, Beacon TX (UART bus) \\
        EDC\_UART\_TX/RX & H1-25           & OBDH, EDC 1, EDC 2   & OBDH TX, EDCs RX (UART bus) \\
        EDC\_UART\_RX/TX & H1-27           & OBDH, EDC 1, EDC 2   & OBDH RX, EDCs TX (UART bus) \\
        EDC\_1\_EN       & H2-3            & OBDH, EDC 1          & EDC 1 enable signal \\
        EDC\_2\_EN       & H2-4            & OBDH, EDC 2          & EDC 2 enable signal \\
        PLX\_EN          & H1-18           & OBDH, Payload X      & Payload X enable (GPIO) \\
        PL\_I2C\_SDA     & H1-41           & OBDH, Payload X      & SDA signal of the payload I2C bus \\
        PL\_I2C\_SCL     & H1-43           & OBDH, Payload X      & SCL signal of the payload I2C bus \\
        GPIO\_N          & H2-8, H2-18, H1-20, H2-20, H2-22, H2-42  & OBDH                 & GPIO pin (not used) \\
        \bottomrule[1.5pt]
    \end{tabular}
    \caption{PC-104 bus signal description.}
    \label{tab:pc104-signals}
\end{table}

\section{Telecommunication}

\begin{landscape}
    \begin{table}[ht]
        \centering
        \begin{tabular}{llcccc}
            \toprule[1.5pt]
            \multirow{2}{*}{\textit{Link}} & \multirow{2}{*}{\textit{Packet Name}} & \multicolumn{4}{c}{\textit{Payload}} \\
            \cmidrule{3-6}
                                      &                       & \textit{ID}  & \textit{Source Callsign}   & \textit{Data (up to 220 bytes)}            & \textit{Size (bytes)} \\
            \midrule
            \multirow{4}{*}{Beacon}   & EPS Data              & 00h & \multirow{4}{*}{``0'' + ``PY0EGU''} & EPS + TTC data                             & 58           \\
                                      & TTC Data              & 01h &                                     & TTC data                                   & 18           \\
                                      & EPS Data              & 02h &                                     & EPS + TTC data                             & 39           \\
                                      & TTC Data              & 03h &                                     & TTC data                                   & 18           \\
            \midrule
            \multirow{5}{*}{Downlink} & Telemetry             & 10h & \multirow{5}{*}{``0'' + ``PY0EGU''} & Flags + OBDH/EPS data                      & 220          \\
                                      & Ping Answer           & 11h &                                     & Requester callsign                         & 15           \\
                                      & Data Request Answer   & 12h &                                     & Req. callsign + data                       & 15 to 155    \\
                                      & Hibernation Feedback  & 13h &                                     & Req. callsign + hibernation in hours       & 17           \\
                                      & Message Broadcast     & 14h &                                     & Req. + dst. callsign + message             & 22 to 60     \\
            \midrule
            \multirow{4}{*}{Uplink}   & Ping Request          & 20h & \multirow{4}{*}{Any Callsign}       & None                                       & 8            \\
                                      & Data Request          & 21h &                                     & Data flags + count + origin + offset       & 16           \\
                                      & Hibernation Request   & 22h &                                     & Req. callsign + hibernation in hours + key & 29           \\
                                      & Broadcast Message     & 23h &                                     & Dst. callsign + message                    & 15 to 46     \\
            \bottomrule[1.5pt]
        \end{tabular}
        \caption{Telecommunication packets and their content.}
        \label{tab:packets-struct}
    \end{table}
\end{landscape}
