%
% introduction.tex
%
% Copyright (C) 2021 by SpaceLab.
%
% GOLDS-UFSC Documentation
%
% This work is licensed under the Creative Commons Attribution-ShareAlike 4.0
% International License. To view a copy of this license,
% visit http://creativecommons.org/licenses/by-sa/4.0/.
%

%
% \brief Introduction chapter.
%
% \author Gabriel Mariano Marcelino <gabriel.mm8@gmail.com>
%
% \institution Universidade Federal de Santa Catarina (UFSC)
%
% \version 0.1.0
%
% \date 2020/06/05
%

\chapter{Introduction} \label{ch:introduction}

GOLDS\nomenclature{\textbf{GOLDS}}{\textit{Global Open Collecting Data System}} stands for Global Open Collecting Data System...

INPE\nomenclature{\textbf{INPE}}{\textit{Instituto Nacional de Pesquisas Espaciais.}}

LIT\nomenclature{\textbf{LIT}}{\textit{Laboratório de Integração e Testes.}}

PCB\nomenclature{\textbf{PCB}}{\textit{Printed Circuit Board.}}

\section{Mission Description}

.

\section{Mission Objectives}

\begin{enumerate}
    \item To serve as a host platform for the EDC payload.
    \item Validate the EDC payload in orbit.
    \item Validate EDC functionality in orbit.
    \item Validate core-satellite functions in orbit.
    \item Evaluate the behavior of the core modules.
    \item Perform experiments on radiation effects in electronic components in orbit.
    \item Serve as relay for amateur radio communications.
\end{enumerate}

\section{Mission Patch}

The mission patch of the GOLDS-UFSC can be seen in \autoref{fig:mission-patch}, it is inspired by the FloripaSat-I patch \cite{floripasat}.

\begin{figure}[!ht]
    \begin{center}
        \includegraphics[width=0.5\textwidth]{figures/golds-ufsc-patch.png}
        \caption{GOLDS-UFSC mission patch.}
        \label{fig:mission-patch}
    \end{center}
\end{figure}
