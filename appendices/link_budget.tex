%
% link_budget.tex
%
% Copyright (C) 2021 by SpaceLab.
%
% GOLDS-UFSC Documentation
%
% This work is licensed under the Creative Commons Attribution-ShareAlike 4.0
% International License. To view a copy of this license,
% visit http://creativecommons.org/licenses/by-sa/4.0/.
%

%
% \brief Link budget calculation appendix.
%
% \author Gabriel Mariano Marcelino <gabriel.mm8@gmail.com>
%
% \institution Universidade Federal de Santa Catarina (UFSC)
%
% \version 0.1.0
%
% \date 2021/02/06
%

\chapter{Link Budget Calculation} \label{anx:link-budget}

\section{Distance to Satellite at Horizon}

\begin{itemize}
    \item $R_{e}$ = Earth radius = 6378 km
    \item $h$ = Satellite altitude = 600 km
    \item $d$ = Distance to sattellite at horizon
\end{itemize}

\begin{equation}
d = \sqrt{2\cdot R_{e}\cdot h + h^{2}} = \sqrt{2\cdot 6378\cdot 600 + 600^{2}} = \mathbf{2830\ km}
\end{equation}


%\section{Power at Receiver}

%\begin{equation}
%P_{r} = P_{t} + G_{t} + G_{r} - L_{p}\ [dB]
%\end{equation}


\section{Free-Space Path Loss}

\begin{equation}
FSPL = \left( \frac{4\pi d f}{c} \right)^{2}
\end{equation}

\begin{equation}
    \begin{split}
        FSPL^{dB} & = 20\log\left(\frac{4\pi}{c}\right) + 20\log\left(d\right) + 20\log\left(f\right) \\
                  & = 32,45 + 20\log\left(\frac{d}{1\ km}\right) + 20\log\left(\frac{f}{1\ MHz}\right) \\
    \end{split}
\end{equation}

\subsection{Beacon}

\begin{equation}
FSPL^{dB}_{max} = 32,45 + 20\log\left(\frac{2830}{1\ km}\right) + 20\log\left(\frac{145,97}{1\ MHz}\right) = \mathbf{144,8\ dB}
\end{equation}

\begin{equation}
FSPL^{dB}_{min} = 32,45 + 20\log\left(\frac{600}{1\ km}\right) + 20\log\left(\frac{145,97}{1\ MHz}\right) = \mathbf{131,3\ dB}
\end{equation}

\begin{equation}
\mathbf{131,3 \leq FSPL^{dB} \leq 144,8\ dB}
\end{equation}

\subsection{Downlink/Uplink}

\begin{equation}
FSPL_{max} = 32,45 + 20\cdot \log\left(\frac{2830}{1\ km}\right) + 20\cdot \log\left(\frac{436,9}{1\ MHz}\right) = \mathbf{154,3\ dB}
\end{equation}

\begin{equation}
FSPL_{min} = 32,45 + 20\cdot \log\left(\frac{600}{1\ km}\right) + 20\cdot \log\left(\frac{436,9}{1\ MHz}\right) = \mathbf{140,8\ dB}
\end{equation}

\begin{equation}
\mathbf{140,8 \leq L_{p} \leq 154,3\ dB}
\end{equation}

\subsection{Uplink (Payload)}

.

\section{Signal-to-Noise-Ratio}

The Signal-to-Noise-Ratio (SNR\nomenclature{\textbf{SNR}}{\textit{Signal To Noise Ratio}}) of the beacon signal can be expressed using the \autoref{eq:snr}:

\begin{equation} \label{eq:snr}
SNR = \frac{E_{b}}{N_{0}} = \frac{P_{t}G_{t}G_{r}}{kT_{s}RL_{p}}
\end{equation}

Where:

\begin{itemize}
    \item $P_{t}$ = Transmitter power
    \item $G_{t}$ = Transmitter gain
    \item $G_{r}$ = Receiver gain
    \item $k$ = Boltzmann's constant ($\cong 1,3806 \times 10^{-23}\ J/K$)
    \item $T_{s}$ = System noise temperature
    \item $R$ = Data rate in bits per seconds (bps)
    \item $L_{p}$ = Free-Space Path Loss (FSPL)
\end{itemize}

The SNR value in decibels can be calculated using the \autoref{eq:snr-db}:

\begin{equation} \label{eq:snr-db}
    \begin{split}
        SNR^{dB} & = 10\log_{10}\left( \frac{E_{b}}{N_{0}} \right) = 10\log_{10} \left( \frac{P_{t}G_{t}G_{r}}{kT_{s}RL_{p}} \right) \\
                 & = P_{t}^{dBm} - 30 + G_{t}^{dBi} + G_{r}^{dBi} - L_{p}^{dB} - 10\log k - 10\log T_{s} - 10\log R
    \end{split}
\end{equation}

\subsection{Beacon}

\begin{equation}
SNR^{dB}_{max} = 30 - 30 + 0 + 15 - 131,3 + 228,6 - 30,64 - 30,79 = 50,87\ dB
\end{equation}

\begin{equation}
SNR^{dB}_{min} = 30 - 30 + 0 + 15 - 144,8 + 228,6 -30,64 -30,79 = 37,37\ dB
\end{equation}

\begin{equation}
\mathbf{37,37 \leq SNR^{dB} \leq 50,87\ dB}
\end{equation}

\subsection{Downlink/Uplink}

.

\subsection{Uplink (Payload)}

.
